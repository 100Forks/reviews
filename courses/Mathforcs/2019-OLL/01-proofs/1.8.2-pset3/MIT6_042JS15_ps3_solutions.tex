\documentclass[14pt]{extarticle} 
\usepackage{amsmath,mathtools,amsfonts,amsthm,amssymb,hyperref}
\usepackage{wasysym,geometry,bussproofs,latexsym,parskip,bookmark}
\newtheorem{defn}{Definition}
\newtheorem{thm}{Theorem}
\newtheorem{claim}{Claim}
\newtheorem{lemma}{Lemma}
\hypersetup{colorlinks,allcolors=blue,linktoc=all}
\geometry{a4paper} 
\geometry{margin=0.5in}
\title{Math for CS 2019 Problem Set 3 solutions}
\author{https://github.com/spamegg1}
\begin{document}
\maketitle
\tableofcontents

\section{Problem 1}
The Fibonacci numbers $F_0, F_1, F_2, \ldots$ are defined as follows:

\begin{displaymath}
   F_n \Coloneqq \left\{
     \begin{array}{lr}
       0 & \text{ if } n = 0\\
       1 & \text{ if } n = 1\\
       F_{n-1} + F_{n-2} & \text{ if } n > 1\\
     \end{array}
   \right.
\end{displaymath} 
Prove, using Strong Induction, the following closed-form formula for $F_n$:
$$
F_n = \frac{p^n - q^n}{\sqrt{5}}
$$
where $p = \frac{1 + \sqrt{5}}{2}$ and $q = \frac{1 - \sqrt{5}}{2}$.

\textit{Hint:} Note that $p$ and $q$ are the roots of $x^2 - x - 1 = 0$, so $p^2 = p+1$ and $q^2 = q+1$.
\begin{proof}
For all natural numbers $n$ let $P(n)$ be the statement: $F_n = \frac{p^n - q^n}{\sqrt{5}}$.

There are two base cases to prove. First $P(0)$.

$F_0 = 0$, by definition of $F_n$. 
$$
\frac{p^0 - q^0}{\sqrt{5}} = \frac{1 - 1}{\sqrt{5}} = 0
$$
So $\displaystyle F_0 = \frac{p^0 - q^0}{\sqrt{5}}$, which means $P(0)$ is true. Next is $P(1)$.

$F_1 = 1$, by definition of $F_n$. 
$$
\frac{p^1 - q^1}{\sqrt{5}} = \frac{\frac{1 + \sqrt{5}}{2} - \frac{1 - \sqrt{5}}{2}}{\sqrt{5}} = \frac{\frac{1 + \sqrt{5} - 1 + \sqrt{5}}{2}}{\sqrt{5}} = \frac{\frac{2\sqrt{5}}{2}}{\sqrt{5}} = 1
$$
So $\displaystyle F_1 = \frac{p^1 - q^1}{\sqrt{5}}$, which means $P(1)$ is true.

Now the Strong Induction step. Assume $n > 1$ and assume that for all natural numbers $k < n$, $P(k)$ is true. We want to prove $P(n)$ is true.

By the induction hypothesis, $P(n - 2)$ is true and $P(n - 1)$ is true. So:
$$
F_{n-2} = \frac{p^{n-2} - q^{n-2}}{\sqrt{5}}
$$
and
$$
F_{n-1} = \frac{p^{n-1} - q^{n-1}}{\sqrt{5}}
$$
We want to prove $\displaystyle F_{n} = \frac{p^{n} - q^{n}}{\sqrt{5}}$.
$$
\begin{array}{cccc}
F_n & = & F_{n-1} + F_{n-2} & \text{(by definition of } F_n) \\
 & = & \displaystyle \frac{p^{n-1} - q^{n-1}}{\sqrt{5}} + \frac{p^{n-2} - q^{n-2}}{\sqrt{5}}& \text{(by induction hypothesis)}\\
 & = & \displaystyle \frac{(p^{n-1} + p^{n-2}) - (q^{n-1} + q^{n-2})}{\sqrt{5}} & \\
 & = & \displaystyle \frac{p^{n-2}(p + 1) - q^{n-2}(q + 1)}{\sqrt{5}} & \\
 & = & \displaystyle \frac{p^{n-2}(p^2) - q^{n-2}(q^2)}{\sqrt{5}} & \text{(by the properties of }p, q) \\
 & = &\displaystyle \frac{p^{n} - q^{n}}{\sqrt{5}} & 
\end{array}
$$
This proves $P(n)$. Therefore by the Principle of Strong Induction $P(n)$ is true for all natural numbers $n$.
\end{proof}

\section{Problem 2}
The Block Stacking Game goes as follows: 

You begin with a stack of $n$ boxes. 

Then you make a sequence of moves. 

In each move, you divide one stack of boxes into two nonempty stacks. 

The game ends when you have $n$ stacks, each containing a single box.

You earn points for each move; in particular, if you divide one stack of height $a + b$ into two stacks with heights $a$ and $b$, then you score $ab$ points for that move.

Your overall score is the sum of the points that you earn for each move. What strategy should you use to maximize your total score?

Define the potential, $p(S)$, of a stack of blocks, $S$, to be $k(k - 1)/2$ where $k$ is the number of blocks in $S$. 

Define the potential, $p(A)$, of a set of stacks, $A$, to be the sum of the potentials of the stacks in $A$.

Show that for any set of stacks, $A$, if a sequence of moves starting with $A$ leads to another set of stacks, $B$, then $p(A) \geq p(B)$, and the score for this sequence of moves is $p(A) - p(B)$.

\textit{Hint:} Try induction on the number of moves to get from $A$ to $B$.
\begin{proof}
Intuitively, the best strategy is to divide stacks as evenly (in half) as possible, to maximize the product $ab$ in each move.

Let us try to formulate the problem so that it can be solved by induction using the hint.

For a natural number $n$, define $Q(n)$ to be the statement: 

``For any set of stacks $A$, and for any set of stacks $B$ that can be reached from $A$ by a sequence of $n$ moves, $p(A) \geq p(B)$, and the score for this sequence of moves is $p(A) - p(B)$.''

We want to prove that $Q(n)$ is true for all natural numbers $n$.

\textbf{Base Case.} We need to prove $Q(0)$. Assume that $A, B$ are any sets of stacks where $B$ can be reached from $A$ by a sequence of 0 moves.

Notice that, in this case, we make 0 moves, so $A$ and $B$ are the same set of stacks. Therefore $p(A) = p(B)$, and the score we get is 0, which is equal to $p(A) - p(B)$. So we see that $Q(0)$ is true.

\textbf{Induction Step.} Now assume $n \geq 0$ and assume that $Q(n)$ is true, we want to prove that $Q(n+1)$ is true.

Assume $A,B$ are any sets of stacks where $B$ can be reached from $A$ by a sequence of $n+1$ moves $m_1, m_2, \ldots, m_{n+1}$.

Define $C$ to be the set of stacks that is reached from $A$ by making the sequence of moves $m_1, m_2, \ldots, m_{n}$. (Then $B$ is reached from $C$ by making one more move $m_{n+1}$.)

By the induction hypothesis, $p(A) \geq p(C)$ and the score for the sequence of moves $m_1, m_2, \ldots, m_{n}$ is $p(A) - p(C)$.

For an individual stack $S$, define $s(S)$ (size) to be the number of blocks in that stack.

Define $C_1, \ldots, C_k$ to be all the individual stacks of $C$.

Then the move $m_{n+1}$ splits \textit{one of these stacks} to reach $B$. For the sake of keeping the notation simple, say, without loss of generality, that it splits the stack $C_1$. (Otherwise it would be some stack $C_i$ for some $i \in \{1, \ldots, k\}$, but we can simply swap the places of $C_1$ and $C_i$. This does not affect the potential, or the score.) Say the move $m_{n+1}$ splits stack $C_1$, which has size, say, $a + b$, into two stacks of size $a$ and $b$.

For our proof we will have to compare $p(C)$ to $p(B)$, so we need to compute both. First let's take a look at the potential of $C$. Remember that we are assuming stack $C_1$ has size $a+b$:
$$
p(C) = p(C_1) + \ldots + p(C_k) = \frac{(a+b)(a+b-1)}{2} + \sum_{j = 2}^k \frac{s(C_j)(s(C_j) - 1)}{2}
$$

Let us try to calculate $p(B)$. The stacks of $B$ are the same as the stacks $C_1, \ldots, C_k$ of $C$, with the exception of stack $C_1$, which has been split into two stacks of size $a$ and $b$. So:
$$
p(B) = \frac{a(a-1)}{2} + \frac{b(b-1)}{2} + \sum_{j = 2}^k \frac{s(C_j)(s(C_j) - 1)}{2}
$$

To prove $Q(n+1)$ we need to show that $p(A) \geq p(B)$ and the score for the sequence of moves $m_1, \ldots, m_{n+1}$ is equal to $p(A) - p(B)$. Let's start with the second one.

By the induction hypothesis, the score for the sequence of moves $m_1, \ldots, m_n$ is $p(A) - p(C)$.

The score for the move $m_{n+1}$ is $ab$.

So, the score for the sequence of moves $m_1, \ldots, m_{n+1}$ is $p(A) - p(C) + ab$.

So we need to prove that $p(A) - p(B) = p(A) - p(C) + ab$. 

This is equivalent to proving that $p(C) - p(B) = ab$.

$$
\begin{array}{cccc}
p(C) - p(B) & = &\displaystyle \left( \frac{(a+b)(a+b-1)}{2} + \sum_{j = 2}^k \frac{s(C_j)(s(C_j) - 1)}{2}\right)\\
 &   &\displaystyle - \left(\frac{a(a-1)}{2} + \frac{b(b-1)}{2} + \sum_{j = 2}^k \frac{s(C_j)(s(C_j) - 1)}{2}\right) \\
 & = & \displaystyle \frac{(a+b)(a+b-1)}{2} - \frac{a(a-1)}{2} - \frac{b(b-1)}{2}\\
 & = &\displaystyle \frac{1}{2}((a + b)(a + b - 1) - a(a-1) - b(b-1)) \\
 & = &\displaystyle \frac{1}{2}(a^2 + ab - a + ba + b^2 - b - (a^2-a) - (b^2-b))\\
 & = &\displaystyle \frac{1}{2}(a^2 + 2ab - a + b^2 - b - a^2+a - b^2+b)\\
 & = &\displaystyle \frac{1}{2}\,(2ab)\\
 & = &\displaystyle ab\\
\end{array}
$$

Now all that's left to prove is $p(A) \geq p(B)$. 

By the induction hypothesis $p(A) \geq p(C)$. 

By the above, we know that $p(C) = p(B) + ab$. We also know that $ab > 0$. 

So $p(A) \geq p(C) = p(B) + ab \geq p(B)$, which proves $p(A) \geq p(B)$.

Thus we have proved $Q(n+1)$. So by the Principle of Induction, $Q(n)$ is true for all natural numbers $n$.
\end{proof}

\section{Problem 3}
Let $A$, $B$, and $C$ be sets, and let $f: B \rightarrow C$ and $g: A \rightarrow B$ be functions. 

Let $h: A \rightarrow C$ be the composition, $f \circ g$, that is, $h(x) = f(g(x))$ for $x \in A$. 

Prove or disprove the following claims:

\textit{Hint:} Arguments based on “arrows” using Definition 4.4.2 are fine.

\subsection{(a)}
If $h$ is surjective, then $f$ must be surjective.
\begin{proof}
This is true. Let's prove it.

Assume $h$ is surjective. By definition of surjectivity, for all $c \in C$ there exists $a \in A$ such that $h(a) = c$.

We want to prove that $f$ is surjective. We want to prove that for all $c \in C$ there exists $b \in B$ such that $f(b) = c$.

Assume $c \in C$. There exists $a \in A$ such that $h(a) = c$. By definition of $h$, we know that $h(a) = f(g(a))$. So $g(a)$ must be defined. (Note that $g$ does not have to be total for this to be true!)

Then define $b = g(a) \in B$. So $f(b) = f(g(a)) = c$. So we have shown that there exists $b \in B$ such that $f(b) = c$. This proves that $f$ is surjective.
\end{proof}

\subsection{(b)}
If $h$ is surjective, then $g$ must be surjective.
\begin{proof}
This is false. Let's give a counterexample. It would be much easier to explain with pictures, but it's difficult for me to draw pictures in \LaTeX.

Define sets $A = \{a\}$, $B = \{b_1, b_2\}$ and $C = \{c\}$.

Define $g: A \rightarrow B$ by $g(a) = b_1$. Notice that $g$ is not surjective.

Define $f : B \rightarrow C$ by $f(b_1) = c$ and $f(b_2) = c$.

With these definitions, we see that $h(a) = f(g(a)) = f(b_1) = c$.

Notice that $h(x) = f(g(x))$ is surjective, because for every element $y \in C$, there is an element $x \in A$ such that $h(x) = y$. (In fact $h$ is bijective, both $A$ and $C$ have exactly 1 element, and $h(a) = c$.)
\end{proof}

\subsection{(c)}
If $h$ is injective, then $f$ must be injective.
\begin{proof}
This is false. The same counterexample from part (b) works. $h$ is bijective in that example, but $f$ is not injective, because $f(b_1) = f(b_2)$. (To put it in Prof. Meyer's terms, $f$ does not have the ``$\leq 1$ arrow-in property'' because there are two arrows coming in to $c$.)
\end{proof}

\subsection{(d)}
If $h$ is injective and $f$ is total, then $g$ must be injective.
\begin{proof}
This is true. Let's prove it.

Assume $h$ is injective and $f$ is total. This means:

For all $c \in C$ there is at most one $a \in A$ such that $h(a) = c$, and

for all $b \in B$, $f(b)$ is defined.

We want to prove that $g$ is injective, in other words for all $b \in B$ there is at most one $a \in A$ such that $g(a) = b$.

Argue by contradiction and assume that there exists $b \in B$ such that there are at least two $a_1, a_2 \in A, a_1 \neq a_2$ such that $g(a_1) = b$ and $g(a_2) = b$.

We know that $f(b)$ is defined. It's an element of $C$, let's say $f(b) = c \in C$.

So $h(a_1) = f(g(a_1)) = f(b) = c$ and $h(a_2) = f(g(a_2)) = f(b) = c$. This contradicts the fact that $h$ is injective!

Thus our assumption was false, therefore $g$ is injective.
\end{proof}

\end{document}